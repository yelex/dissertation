\chapter*{Заключение}                       % Заголовок
\addcontentsline{toc}{chapter}{Заключение}  % Добавляем его в оглавление

%% Согласно ГОСТ Р 7.0.11-2011:
%% 5.3.3 В заключении диссертации излагают итоги выполненного исследования, рекомендации, перспективы дальнейшей разработки темы.
%% 9.2.3 В заключении автореферата диссертации излагают итоги данного исследования, рекомендации и перспективы дальнейшей разработки темы.
%% Поэтому имеет смысл сделать эту часть общей и загрузить из одного файла в автореферат и в диссертацию:

Основные результаты работы заключаются в следующем.

Во-первых, цены онлайн-ритейлеров обладают меньшей жесткостью по сравнению с результатами, полученными в работе \cite{cavallo2018scraped} на данных по США и четырем странам Латинской Америки. Это различие обусловлено особенностями нашей выборки, в которой преобладают продовольственные товары с гибкими ценами.

Во-вторых, данные о ценах онлайн-ритейлеров, использованные в исследовании, охватывают неоднородный временной период, что позволило выявить особенности ценового поведения в условиях как относительно низкой, так и высокой инфляции. Было установлено, что после обострения геополитической напряженности в феврале 2022 года цены стали более гибкими. Это связано с высокой инфляцией и необходимостью для фирм чаще корректировать цены в условиях повышенной неопределенности. Кроме того, было обнаружено, что в 2022 году инфляция начала значимо влиять на долю товаров с изменяющимися ценами, что подтверждает справедливость выводов моделей ценообразования, зависящего от состояния экономики, и согласуется с выводами работы \cite{gagnon2009}, касающимися периодов низкой и высокой инфляции в мексиканской экономике.

В-третьих, В совокупности, анализируемый набор стилизованных фактов указывает на то, что ценовое поведение российских ритейлеров соответствует гибридным моделям ценообразования, сочетающим элементы как зависящего от состояния экономики, так и зависящего от времени подхода.
%\input{common/concl}
%И какая-нибудь заключающая фраза.
%
%Последний параграф может включать благодарности.  В заключение автор
%выражает благодарность и большую признательность научному руководителю
%Иванову~И.\,И. за поддержку, помощь, обсуждение результатов и~научное
%руководство. Также автор благодарит Сидорова~А.\,А. и~Петрова~Б.\,Б.
%за помощь в~работе с~образцами, Рабиновича~В.\,В. за предоставленные
%образцы и~обсуждение результатов, Занудятину~Г.\,Г. и авторов шаблона
%*Russian-Phd-LaTeX-Dissertation-Template* за~помощь в оформлении
%диссертации. Автор также благодарит много разных людей
%и~всех, кто сделал настоящую работу автора возможной.
