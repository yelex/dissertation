
{\actuality} Одним из ключевых направлений денежно-кредитной политики государства является поддержание инфляции на стабильном и относительно низком уровне. В парадигме новокейнсианской логики, которая лежит в основе моделей прогнозирования большинства центральных банков мира, инфляция имеет строго определенные предпосылки относительно свойств процесса ценообразования. Вместе с тем до недавнего времени исследование этих предпосылок было затруднено, поскольку доступ к большим массивам данных по ценам был ограничен для исследователей.

В начале 2000-х гг. появились первые работы, проливающие свет на характеристики ценообразования на микроуровне. Эти работы, как правило, опирались на неопубликованные данные национальных статистических органов и содержали в себе как ряд достоинств, так и недостатков. Основным достоинством таких данных являлась их репрезентативность с точки зрения покрытия всех категорий потребления из индекса потребительских цен. 

Вместе с тем такие данные не позволяли отслеживать ряд характеристик, важных с точки зрения моделирования процессов ценообразования. К примеру, из-за исключения товара из выборки, не связанного с его исчезновением из продажи, данные давали смещенные оценки относительно средней продолжительности неизменности цен. Кроме того, такие данные имели лишь ежемесячную частоту, что не позволяло дать корректную оценку ценовой жесткости для товаров, чей средний период неизменности составляет менее одного месяца. Наконец, получение таких данных по ценам связано с большими издержками из-за привлечения к сбору данных большого числа сотрудников статистических ведомств.

Рост объемов торговли в интернете привел к возникновению еще одного источника данных о ценах – а именно, данных, собранных в результате веб-скраппинга сайтов онлайн-ритейлеров. Веб-скраппинг – это технология автоматизированного сбора данных, при которой информация с определенных частей сайта записывается в таблицу и затем – в базу данных. Такие данные обладают существенными преимуществами: они покрывают, как правило, сразу несколько тысяч товарных позиций, собираются с гораздо меньшими издержками, чем в случае традиционного сбора данных для нужд построения индекса потребительских цен, а также могут собираться с любой регулярностью, что выгодно отличает этот способ сбора данных от любого «физического» способа. 

Вместе с тем онлайн-данные обладают и некоторыми существенными недостатками. Так, данные онлайн-ритейлеров менее репрезентативны, чем традиционные данные национальных статистических ведомств, поскольку все еще не охватывают существенную часть услуг, а также в целом покрывают менее десяти процентов от розничной торговли. Однако результаты ряда эмпирических работ показывают, что даже несмотря на эти недостатки, данные онлайн-ритейлеров обладают полезностью для целей изучения ценового поведения в целом, и для оценки и прогнозирования инфляции, в частности.

Онлайн-ритейл является динамично развивающейся частью расходов потребителей. Вместе с тем, до сих пор неясно, распространяются ли обнаруженные в традиционных офлайн-ценах закономерности на рынок онлайн-ритейла, поскольку в онлайн-ритейле издержки меню, являющиеся одной из ключевых обоснований существования жесткости цен в новокейнсианской теории, очевидно незначительны. Неясно, меняется ли жесткость цен с течением времени, или остается постоянным значением? Насколько скоординированным является изменение цен? В работе \cite{gagnon2009} было показано, что при переходе 10\%-ного порога инфляции ценообразование на традиционных рынках (офлайн-ритейл) начинает меняться. Возникает вопрос, справедливы ли эти изменения для ситуации в российском онлайн-ритейле – по имеющимся с 2019 года данным мы можем посмотреть это, учитывая всплеск инфляции в 2022 году.

% {\progress}
% Этот раздел должен быть отдельным структурным элементом по
% ГОСТ, но он, как правило, включается в описание актуальности
% темы. Нужен он отдельным структурынм элемементом или нет ---
% смотрите другие диссертации вашего совета, скорее всего не нужен.

{\aim} данной работы является анализ особенностей данных онлайн-ритейлеров для исследования динамики цен.
Для~достижения поставленной цели необходимо было решить следующие {\tasks}:
\begin{enumerate}[beginpenalty=10000] % https://tex.stackexchange.com/a/476052/104425
  \item Произвести обзор теоретической и эмпирической литературы по изучению особенностей динамики цен в онлайн-ритейле в сравнении с поведением цен в традиционном офлайн-ритейле
  \item Описать ценовые тенденции в период пандемии коронавирусной инфекции (резкое сокращение спроса, как оно отразилось на онлайн-ритейле), а также тенденции в период после 24 февраля 2022 года, когда инфляция резко выросла. Описать особенности этих тенденций
  \item Выявить аспекты поведения цен, которые меняются во время указанных выше периодов относительной макроэкономической нестабильности
  \item Обнаружить факторы жесткости цен на рынке онлайн-ритейла 
\end{enumerate}


{\novelty}
\begin{enumerate}[beginpenalty=10000] % https://tex.stackexchange.com/a/476052/104425
  \item Впервые в российской практике используются данные, полученные в результате веб-скраппинга сайтов онлайн-ритейлеров за продолжительный срок
  \item В работе впервые в российской практике проведено систематическое исследование использования данных онлайн-ритейлеров для анализа ценовой динамики и особенностей инфляции
  \item Впервые проведен анализ ценового поведения онлайн-ритейлеров с точки зрения соответствия выводам теоретических моделей ценообразования
  \item Впервые обсуждается изменение характеристик ценового поведения онлайн-ритейлеров в продолжительной временной перспективек, охватывающей стабильные и нестабильные макроэкономические периоды. Показано, как пандемия COVID-19 и резкий рост инфляции после февраля 2022 года повлияли на ценовую жесткость в онлайн-ритейле
  % \item Было выполнено оригинальное исследование \ldots
\end{enumerate}

{\influence} Разработанные методы сбора и обработки данных могут быть использованы государственными органами и аналитическими центрами для улучшения мониторинга инфляции. Результаты исследования могут применяться ЦБ РФ для оперативного реагирования на изменения инфляционных процессов, выявленных с помощью онлайн-данных. Полученные выводы о поведении цен в кризисные периоды могут быть полезны ритейлерам для формирования более эффективных стратегий ценообразования.

{\methods} В качестве общенаучных методов исследования используется анализ и синтез, сравнительный анализ, а также индукция и дедукция. В качестве специальных методов экономического анализа использовались: регрессионный анализ, логит-модели.

{\defpositions}
\begin{enumerate}[beginpenalty=10000] % https://tex.stackexchange.com/a/476052/104425
  \item Показано, что макроэкономические шоки, такие как пандемия COVID-19 и инфляционный всплеск в 2022 году, существенно изменяют поведение цен в онлайн-ритейле.
  \item Данные с ежедневной частотой дают более точные оценки периодов ценовой неизменности и размера изменений, чем традиционные офлайн-методы.
  \item Использование технологий автоматизированного сбора данных обеспечивает высокую скорость и точность мониторинга цен
  \item Данные онлайн-ритейлеров, несмотря на ограниченное покрытие услуг, могут использоваться для точного анализа ценовой динамики и оценки инфляции
\end{enumerate}

{\reliability} полученных результатов обеспечивается использованием общепризнанных экономико-статистических методов, подтверждением на репрезентативных данных, а также их сопоставлением с теоретическими выводами и эмпирическими исследованиями других авторов.


{\probation}
Основные результаты работы докладывались~на:
перечисление основных конференций, симпозиумов и~т.\:п.

{\contribution} Автор принимал активное участие \ldots

\ifnumequal{\value{bibliosel}}{0}
{%%% Встроенная реализация с загрузкой файла через движок bibtex8. (При желании, внутри можно использовать обычные ссылки, наподобие `\cite{vakbib1,vakbib2}`).
    {\publications} Основные результаты по теме диссертации изложены
    в~XX~печатных изданиях,
    X из которых изданы в журналах, рекомендованных ВАК,
    X "--- в тезисах докладов.
}%
{%%% Реализация пакетом biblatex через движок biber
    \begin{refsection}[bl-author, bl-registered]
        % Это refsection=1.
        % Процитированные здесь работы:
        %  * подсчитываются, для автоматического составления фразы "Основные результаты ..."
        %  * попадают в авторскую библиографию, при usefootcite==0 и стиле `\insertbiblioauthor` или `\insertbiblioauthorgrouped`
        %  * нумеруются там в зависимости от порядка команд `\printbibliography` в этом разделе.
        %  * при использовании `\insertbiblioauthorgrouped`, порядок команд `\printbibliography` в нём должен быть тем же (см. biblio/biblatex.tex)
        %
        % Невидимый библиографический список для подсчёта количества публикаций:
        \printbibliography[heading=nobibheading, section=1, env=countauthorvak,          keyword=biblioauthorvak]%
        \printbibliography[heading=nobibheading, section=1, env=countauthorwos,          keyword=biblioauthorwos]%
        \printbibliography[heading=nobibheading, section=1, env=countauthorscopus,       keyword=biblioauthorscopus]%
        \printbibliography[heading=nobibheading, section=1, env=countauthorconf,         keyword=biblioauthorconf]%
        \printbibliography[heading=nobibheading, section=1, env=countauthorother,        keyword=biblioauthorother]%
        \printbibliography[heading=nobibheading, section=1, env=countregistered,         keyword=biblioregistered]%
        \printbibliography[heading=nobibheading, section=1, env=countauthorpatent,       keyword=biblioauthorpatent]%
        \printbibliography[heading=nobibheading, section=1, env=countauthorprogram,      keyword=biblioauthorprogram]%
        \printbibliography[heading=nobibheading, section=1, env=countauthor,             keyword=biblioauthor]%
        \printbibliography[heading=nobibheading, section=1, env=countauthorvakscopuswos, filter=vakscopuswos]%
        \printbibliography[heading=nobibheading, section=1, env=countauthorscopuswos,    filter=scopuswos]%
        %
        \nocite{*}%
        %
        {\publications} Основные результаты по теме диссертации изложены в~\arabic{citeauthor}~печатных изданиях,
        \arabic{citeauthorvak} из которых изданы в журналах, рекомендованных ВАК\sloppy%
        \ifnum \value{citeauthorscopuswos}>0%
            , \arabic{citeauthorscopuswos} "--- в~периодических научных журналах, индексируемых Web of~Science и Scopus\sloppy%
        \fi%
        \ifnum \value{citeauthorconf}>0%
            , \arabic{citeauthorconf} "--- в~тезисах докладов.
        \else%
            .
        \fi%
        \ifnum \value{citeregistered}=1%
            \ifnum \value{citeauthorpatent}=1%
                Зарегистрирован \arabic{citeauthorpatent} патент.
            \fi%
            \ifnum \value{citeauthorprogram}=1%
                Зарегистрирована \arabic{citeauthorprogram} программа для ЭВМ.
            \fi%
        \fi%
        \ifnum \value{citeregistered}>1%
            Зарегистрированы\ %
            \ifnum \value{citeauthorpatent}>0%
            \formbytotal{citeauthorpatent}{патент}{}{а}{}\sloppy%
            \ifnum \value{citeauthorprogram}=0 . \else \ и~\fi%
            \fi%
            \ifnum \value{citeauthorprogram}>0%
            \formbytotal{citeauthorprogram}{программ}{а}{ы}{} для ЭВМ.
            \fi%
        \fi%
        % К публикациям, в которых излагаются основные научные результаты диссертации на соискание учёной
        % степени, в рецензируемых изданиях приравниваются патенты на изобретения, патенты (свидетельства) на
        % полезную модель, патенты на промышленный образец, патенты на селекционные достижения, свидетельства
        % на программу для электронных вычислительных машин, базу данных, топологию интегральных микросхем,
        % зарегистрированные в установленном порядке.(в ред. Постановления Правительства РФ от 21.04.2016 N 335)
    \end{refsection}%
    \begin{refsection}[bl-author, bl-registered]
        % Это refsection=2.
        % Процитированные здесь работы:
        %  * попадают в авторскую библиографию, при usefootcite==0 и стиле `\insertbiblioauthorimportant`.
        %  * ни на что не влияют в противном случае
        \nocite{vakbib2}%vak
        \nocite{patbib1}%patent
        \nocite{progbib1}%program
        \nocite{bib1}%other
        \nocite{confbib1}%conf
    \end{refsection}%
        %
        % Всё, что вне этих двух refsection, это refsection=0,
        %  * для диссертации - это нормальные ссылки, попадающие в обычную библиографию
        %  * для автореферата:
        %     * при usefootcite==0, ссылка корректно сработает только для источника из `external.bib`. Для своих работ --- напечатает "[0]" (и даже Warning не вылезет).
        %     * при usefootcite==1, ссылка сработает нормально. В авторской библиографии будут только процитированные в refsection=0 работы.
}
